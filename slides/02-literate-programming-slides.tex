\PassOptionsToPackage{unicode=true}{hyperref} % options for packages loaded elsewhere
\PassOptionsToPackage{hyphens}{url}
%
\documentclass[ignorenonframetext,]{beamer}
\setbeamertemplate{caption}[numbered]
\setbeamertemplate{caption label separator}{: }
\setbeamercolor{caption name}{fg=normal text.fg}
\beamertemplatenavigationsymbolsempty
\usepackage{lmodern}
\usepackage{amssymb,amsmath}
\usepackage{ifxetex,ifluatex}
\usepackage{fixltx2e} % provides \textsubscript
\ifnum 0\ifxetex 1\fi\ifluatex 1\fi=0 % if pdftex
  \usepackage[T1]{fontenc}
  \usepackage[utf8]{inputenc}
  \usepackage{textcomp} % provides euro and other symbols
\else % if luatex or xelatex
  \usepackage{unicode-math}
  \defaultfontfeatures{Ligatures=TeX,Scale=MatchLowercase}
\fi
% use upquote if available, for straight quotes in verbatim environments
\IfFileExists{upquote.sty}{\usepackage{upquote}}{}
% use microtype if available
\IfFileExists{microtype.sty}{%
\usepackage[]{microtype}
\UseMicrotypeSet[protrusion]{basicmath} % disable protrusion for tt fonts
}{}
\IfFileExists{parskip.sty}{%
\usepackage{parskip}
}{% else
\setlength{\parindent}{0pt}
\setlength{\parskip}{6pt plus 2pt minus 1pt}
}
\usepackage{hyperref}
\hypersetup{
            pdftitle={Literate Programming with R},
            pdfauthor={Tim Dennis},
            pdfborder={0 0 0},
            breaklinks=true}
\urlstyle{same}  % don't use monospace font for urls
\newif\ifbibliography
\usepackage{color}
\usepackage{fancyvrb}
\newcommand{\VerbBar}{|}
\newcommand{\VERB}{\Verb[commandchars=\\\{\}]}
\DefineVerbatimEnvironment{Highlighting}{Verbatim}{commandchars=\\\{\}}
% Add ',fontsize=\small' for more characters per line
\newenvironment{Shaded}{}{}
\newcommand{\AlertTok}[1]{\textcolor[rgb]{1.00,0.00,0.00}{\textbf{#1}}}
\newcommand{\AnnotationTok}[1]{\textcolor[rgb]{0.38,0.63,0.69}{\textbf{\textit{#1}}}}
\newcommand{\AttributeTok}[1]{\textcolor[rgb]{0.49,0.56,0.16}{#1}}
\newcommand{\BaseNTok}[1]{\textcolor[rgb]{0.25,0.63,0.44}{#1}}
\newcommand{\BuiltInTok}[1]{#1}
\newcommand{\CharTok}[1]{\textcolor[rgb]{0.25,0.44,0.63}{#1}}
\newcommand{\CommentTok}[1]{\textcolor[rgb]{0.38,0.63,0.69}{\textit{#1}}}
\newcommand{\CommentVarTok}[1]{\textcolor[rgb]{0.38,0.63,0.69}{\textbf{\textit{#1}}}}
\newcommand{\ConstantTok}[1]{\textcolor[rgb]{0.53,0.00,0.00}{#1}}
\newcommand{\ControlFlowTok}[1]{\textcolor[rgb]{0.00,0.44,0.13}{\textbf{#1}}}
\newcommand{\DataTypeTok}[1]{\textcolor[rgb]{0.56,0.13,0.00}{#1}}
\newcommand{\DecValTok}[1]{\textcolor[rgb]{0.25,0.63,0.44}{#1}}
\newcommand{\DocumentationTok}[1]{\textcolor[rgb]{0.73,0.13,0.13}{\textit{#1}}}
\newcommand{\ErrorTok}[1]{\textcolor[rgb]{1.00,0.00,0.00}{\textbf{#1}}}
\newcommand{\ExtensionTok}[1]{#1}
\newcommand{\FloatTok}[1]{\textcolor[rgb]{0.25,0.63,0.44}{#1}}
\newcommand{\FunctionTok}[1]{\textcolor[rgb]{0.02,0.16,0.49}{#1}}
\newcommand{\ImportTok}[1]{#1}
\newcommand{\InformationTok}[1]{\textcolor[rgb]{0.38,0.63,0.69}{\textbf{\textit{#1}}}}
\newcommand{\KeywordTok}[1]{\textcolor[rgb]{0.00,0.44,0.13}{\textbf{#1}}}
\newcommand{\NormalTok}[1]{#1}
\newcommand{\OperatorTok}[1]{\textcolor[rgb]{0.40,0.40,0.40}{#1}}
\newcommand{\OtherTok}[1]{\textcolor[rgb]{0.00,0.44,0.13}{#1}}
\newcommand{\PreprocessorTok}[1]{\textcolor[rgb]{0.74,0.48,0.00}{#1}}
\newcommand{\RegionMarkerTok}[1]{#1}
\newcommand{\SpecialCharTok}[1]{\textcolor[rgb]{0.25,0.44,0.63}{#1}}
\newcommand{\SpecialStringTok}[1]{\textcolor[rgb]{0.73,0.40,0.53}{#1}}
\newcommand{\StringTok}[1]{\textcolor[rgb]{0.25,0.44,0.63}{#1}}
\newcommand{\VariableTok}[1]{\textcolor[rgb]{0.10,0.09,0.49}{#1}}
\newcommand{\VerbatimStringTok}[1]{\textcolor[rgb]{0.25,0.44,0.63}{#1}}
\newcommand{\WarningTok}[1]{\textcolor[rgb]{0.38,0.63,0.69}{\textbf{\textit{#1}}}}
\usepackage{graphicx,grffile}
\makeatletter
\def\maxwidth{\ifdim\Gin@nat@width>\linewidth\linewidth\else\Gin@nat@width\fi}
\def\maxheight{\ifdim\Gin@nat@height>\textheight\textheight\else\Gin@nat@height\fi}
\makeatother
% Scale images if necessary, so that they will not overflow the page
% margins by default, and it is still possible to overwrite the defaults
% using explicit options in \includegraphics[width, height, ...]{}
\setkeys{Gin}{width=\maxwidth,height=\maxheight,keepaspectratio}
% Prevent slide breaks in the middle of a paragraph:
\widowpenalties 1 10000
\raggedbottom
\setbeamertemplate{part page}{
\centering
\begin{beamercolorbox}[sep=16pt,center]{part title}
  \usebeamerfont{part title}\insertpart\par
\end{beamercolorbox}
}
\setbeamertemplate{section page}{
\centering
\begin{beamercolorbox}[sep=12pt,center]{part title}
  \usebeamerfont{section title}\insertsection\par
\end{beamercolorbox}
}
\setbeamertemplate{subsection page}{
\centering
\begin{beamercolorbox}[sep=8pt,center]{part title}
  \usebeamerfont{subsection title}\insertsubsection\par
\end{beamercolorbox}
}
\AtBeginPart{
  \frame{\partpage}
}
\AtBeginSection{
  \ifbibliography
  \else
    \frame{\sectionpage}
  \fi
}
\AtBeginSubsection{
  \frame{\subsectionpage}
}
\setlength{\emergencystretch}{3em}  % prevent overfull lines
\providecommand{\tightlist}{%
  \setlength{\itemsep}{0pt}\setlength{\parskip}{0pt}}
\setcounter{secnumdepth}{0}

% set default figure placement to htbp
\makeatletter
\def\fps@figure{htbp}
\makeatother


\title{Literate Programming with R}
\author{Tim Dennis}
\date{}

\begin{document}
\frame{\titlepage}

\begin{frame}{%
\protect\hypertarget{overview}{%
Overview}}

\begin{itemize}
\tightlist
\item
  Introduction
\item
  Literate programming with Markdown and R
\item
  Explore a working Knitr document
\end{itemize}

\end{frame}

\begin{frame}{%
\protect\hypertarget{motivating-literate-programming}{%
Motivating literate programming}}

(Adapted from Jenny Bryan’s slides -
\url{https://github.com/datacarpentry/rr-literate-programming})

\end{frame}

\begin{frame}{%
\protect\hypertarget{issues}{%
Issues}}

\textbf{How to}

\begin{itemize}
\tightlist
\item
  organize your work?
\item
  make work more pleasant for yourself? (less tedium, less manual, less
  \ldots{})
\item
  reduce friction for collaboration?
\item
  reduce friction for communication?
\item
  make your work navigable, interpretable, and repeatable by others?
\end{itemize}

A lot of this can be built into the normal coding and analysis process
by using specific tools and habits.

\note{\begin{itemize}
\tightlist
\item
  Motivating example: a small report written with Word, possibly based
  on last exercise from morning

  \begin{itemize}
  \tightlist
  \item
    E.g., one page report about one country from Gapminder
  \item
    What if you want to change to a different country? Rewrite some
    text, remake a figure, re-insert new figure.
  \item
    This is a drag and error prone. This is the problem R Markdown can
    solve.
  \end{itemize}
\end{itemize}}

\end{frame}

\begin{frame}{%
\protect\hypertarget{getting-the-analysis-right-is-only-one-link}{%
Getting the analysis right is only one link}}

Process, packaging, and presentation are often the weak links in the
chain.

\begin{figure}
\centering
\includegraphics{../media/brokenChain.jpg}
\caption{The weakest link in the chain}
\end{figure}

\end{frame}

\hypertarget{markdown}{%
\section{Markdown}\label{markdown}}

\begin{frame}{%
\protect\hypertarget{what-is-markdown}{%
What is Markdown?}}

\begin{itemize}
\tightlist
\item
  Markdown is a particular type of markup language. Markup languages are
  designed to produce documents from plain text.

  \begin{itemize}
  \tightlist
  \item
    You may be familiar with \emph{LaTeX}, another (though less human
    friendly) text markup language.
  \end{itemize}
\item
  Tools render markdown to different formats (for example,
  HTML/pdf/Word).

  \begin{itemize}
  \tightlist
  \item
    The main tool for rendering markdown is
    \href{http://johnmacfarlane.net/pandoc/}{pandoc}).
  \end{itemize}
\end{itemize}

Adapted from
\href{http://cpsievert.github.io/slides/markdown/\#/2}{Carson Sievert’s
markdown slides}.

\end{frame}

\begin{frame}{%
\protect\hypertarget{why-markdown}{%
Why Markdown?}}

\begin{itemize}
\tightlist
\item
  Easy to learn and use. Focus on \emph{content}, rather than
  \emph{coding} and debugging \emph{errors}.
\item
  Easy to edit yet very flexible. Can add HTML, CSS, and JavaScript.
\item
  The rendering process is through tools and is thus an automatable and
  repeatable process.
\item
  Lends itself well to version control, collaboration, sharing, and
  reuse.
\end{itemize}

Adapted from
\href{http://cpsievert.github.io/slides/markdown/\#/1}{Carson Sievert’s
markdown slides}.

\end{frame}

\begin{frame}{%
\protect\hypertarget{markdown-enables-fast-publishing-to-the-web}{%
Markdown enables fast publishing to the web}}

\textbf{Markdown}: Easy to write and read in an editor

\textbf{HTML}: Easy to publish and read on the web

\end{frame}

\begin{frame}{%
\protect\hypertarget{markdown-versus-html-code}{%
Markdown versus HTML code}}

\end{frame}

\begin{frame}[fragile]{%
\protect\hypertarget{markdown-versus-rendered-html}{%
Markdown versus rendered HTML}}

This is a Markdown document.

Medium header

It’s easy to do \emph{italics} or \textbf{make things bold}.

\begin{quote}
All models are wrong, but some are useful. An approximate answer to the
right problem is worth a good deal more than an exact answer to an
approximate problem.
\end{quote}

Code block below. Just affects formatting here.

\begin{verbatim}
x <- 3 * 4
\end{verbatim}

I can haz equations. Inline equations, such as the average is computed
as \(\frac{1}{n} \sum_{i=1}^{n} x_{i}\). Or display equations like this:

\[
\begin{equation*}
|x|= 
\begin{cases} x & \text{if $x\ge 0$,} \\\\
-x &\text{if $x\lt 0$.}
\end{cases}
\end{equation*}
\]

\end{frame}

\begin{frame}[fragile]{%
\protect\hypertarget{markdown-can-be-rendered-to-multiple-formats}{%
Markdown can be rendered to multiple formats}}

\begin{itemize}
\tightlist
\item
  \texttt{pandoc} is a swiss-army knife tool for conversion
\end{itemize}

\end{frame}

\hypertarget{r-markdown}{%
\section{R Markdown}\label{r-markdown}}

\begin{frame}{%
\protect\hypertarget{r-markdown-is-rendered-to-markdown}{%
R Markdown is rendered to Markdown}}

\end{frame}

\begin{frame}[fragile]{%
\protect\hypertarget{ideas-code-and-generated-results-tied-together}{%
Ideas, code, and generated results tied together}}

This is an R Markdown document.

\begin{Shaded}
\begin{Highlighting}[]
\NormalTok{x <-}\StringTok{ }\KeywordTok{rnorm}\NormalTok{(}\DecValTok{1000}\NormalTok{)}
\KeywordTok{head}\NormalTok{(x)}
\end{Highlighting}
\end{Shaded}

\begin{verbatim}
## [1]  1.0795501 -0.5202478 -0.6306888 -0.3469508  0.6252589  0.6442898
\end{verbatim}

\texttt{knitr} offers a lot of control over representing different types
of output. We can also have inline R expressions computed on the fly.
The mean \(\bar{x} = \frac{1}{n} \sum_{i=1}^{n} x_{i}\) of the 1000
random variates we generated is -0.026. No more copy-paste, including
for figures:

\begin{Shaded}
\begin{Highlighting}[]
\KeywordTok{plot}\NormalTok{(}\KeywordTok{density}\NormalTok{(x))}
\end{Highlighting}
\end{Shaded}

\includegraphics{02-literate-programming-slides_files/figure-beamer/unnamed-chunk-3-1.pdf}

\end{frame}

\begin{frame}{%
\protect\hypertarget{can-be-rendered-interactively-in-rstudio}{%
Can be rendered interactively in RStudio}}

\end{frame}

\begin{frame}[fragile]{%
\protect\hypertarget{rendering-can-be-automated-is-thus-repeatable}{%
Rendering can be automated is thus repeatable}}

From within R:

\begin{Shaded}
\begin{Highlighting}[]
\NormalTok{rmarkdown}\OperatorTok{::}\KeywordTok{render}\NormalTok{(}\StringTok{"filename.Rmd"}\NormalTok{)}
\end{Highlighting}
\end{Shaded}

From the command line:

\begin{Shaded}
\begin{Highlighting}[]
\NormalTok{$ }\ExtensionTok{Rscript}\NormalTok{ -e }\StringTok{"rmarkdown::render('filename.Rmd')"}
\end{Highlighting}
\end{Shaded}

\end{frame}

\begin{frame}[fragile]{%
\protect\hypertarget{summary}{%
Summary}}

\begin{itemize}
\item
  R Markdown enables ideas and questions, the code that implements them,
  and the results generated by the implementation, to all stay together.
\item
  R Markdown toolchain allows automated, repeatable rendering

  \begin{itemize}
  \tightlist
  \item
    for publishing to the web and viewing through a browser,
  \item
    and (through LaTeX) to obtain a submittable manuscript (in PDF or
    Word).
  \end{itemize}
\item
  \texttt{knitr} is not limited to executing R code. See the book
  \emph{Dynamic documents with R and knitr} by Yihui Xie, part of the
  CRC Press / Chapman \& Hall R Series (2013). ISBN:
  \href{http://www.isbnsearch.org/isbn/9781482203530}{9781482203530}.
\end{itemize}

\end{frame}

\end{document}
